\section{Introduzione}
\subsection{Analisi iniziale}
La seguente documentazione specifica nel dettaglio il dominio applicativo d'interesse e serve ad analizzare nel modo più preciso possibile tutti gli aspetti che lo riguardano.

\subsection{Traccia}
La \textbf{Federazione Internazionale dell'Automobile} (FIA) intende tener traccia dei campionati mondiali di Formula 1. Ogni anno si apre con l'inaugurazione del mondiale di cui si vuole tener traccia. Ogni mondiale è caratterizzato dell'edizione, che identifica ogni mondiale, dal regolamento, da una descrizione formale e dal numero di auto partecipanti. In ciascun mondiale partecipano delle scuderie e, per ognuna di esse, si vuole tener traccia: del nome (univoco per ogni scuderia), dell'anno di fondazione e dei team che ci lavorano. Ogni team è caratterizzato dal settore ricoperto il quale, assieme alla scuderia di appartenenza, è in grado d'identificare univocamente ogni team; questo è possibile in quanto non vi possono essere più team all'interno di una scuderia che ricoprono lo stesso settore. In ogni mondiale partecipano al più 10 scuderie e, per ogni scuderia, i team sviluppano 2 vetture. Le vetture sono obbligatoriamente diverse ogni anno e, pertanto, è possibile identificarle in base all'edizione del mondiale e al numero in gara (diverso per ogni vettura). Inoltre, ciascuna vettura, si compone anche di nome e numero di cavalli e, per ogni singola vettura in competizione, potrebbero esse necessari dei commenti. 

Dei dipendenti che lavorano per i team si vuole tener traccia: della matricola, del nome, del cognome e della data di nascita. Fra le tipologie di dipendenti siamo interessati a: Team principal e Piloti. Dei team principal si vuole memorizzare il numero di anni di esperienza mentre, per i piloti, per questioni regolamentari, bisogna conservare anche il BMI (Body Mass Index) composto da peso e l'altezza.
I piloti candidati per le gare otterranno al termine di ogni gara: una posizione finale e il relativo punteggio. I piloti gareggeranno su dei tracciati identificati dal nome e caratterizzati dalla nazione di appartenenza; un pilota può gareggia più volte sullo stesso circuito.